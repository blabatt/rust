\section{Ownership}
Ownership passes when new binding occurs.  This is called a \say{move}.  To avoid owning, use \say{borrowed} references instead. If copy trait is implemented, moves are obviated. \\

\subsection*{Borrowing}
\code{let var1 = \&x;} // immutable borrow \\
\code{let var2 = \&mut x;} // mutable borrow \\
\code{let slice = \&s[6..11];} // slice of string s\\
// pass by [borrowed] reference: \\
\code{let var3 = my\_func( \&concrete );} \\
// Note: spawned threads can't generally \\
// borrow because lifetime in parent is \\
// not guaranteed. \\

\subsection*{Moving}
// Force ownership into closure with \say{move}: \\
\code{let c = move |z| z == x; // (aka a \say{capture})} \\

\subsection*{Dropping}
// \say{drop} forces relinquishment of ownership:
\code{drop(var);} // drop is in \code{std::mem::drop} \\

\subsection*{Lifetimes}

\code{\&\textquotesingle a i32} // basic annotation \\
\code{fn f<\textquotesingle a>(x: \&str, y: \&\textquotesingle a str) -> \&\textquotesingle a str \{} // in sig.
\code{struct s<\textquotesingle a> \{} // on structs \\
\code{\textquotesingle static} // static \\
\subsection*{Lifetime Elision Rules} 
Given sets $P$, $L$, $OL$ for parameters, lifetimes, output lifetime, respectively:
\begin{enumerate}
    \item $\forall P \exists L $ \say{each param. has a lifetime}
    \item $\abs{L} = 1 \implies OL = L(0)$ \\ \say{output lifetime is same as a solitary input lifetime}
    \item $\exists$ \&self $\in P \implies OL = L(self)$ \\ \say{\&self's lifetime taken by default as output lifetime}
\end{enumerate} \